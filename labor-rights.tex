% Options for packages loaded elsewhere
\PassOptionsToPackage{unicode}{hyperref}
\PassOptionsToPackage{hyphens}{url}
\PassOptionsToPackage{dvipsnames,svgnames,x11names}{xcolor}
%
\documentclass[
  12pt,
]{article}
\title{Labor Rights and Global Supply Chains}
\usepackage{etoolbox}
\makeatletter
\providecommand{\subtitle}[1]{% add subtitle to \maketitle
  \apptocmd{\@title}{\par {\large #1 \par}}{}{}
}
\makeatother
\subtitle{An Annotated Bibliography}
\author{Emmanuel Teitelbaum}
\date{June 03, 2022}

\usepackage{amsmath,amssymb}
\usepackage{lmodern}
\usepackage{iftex}
\ifPDFTeX
  \usepackage[T1]{fontenc}
  \usepackage[utf8]{inputenc}
  \usepackage{textcomp} % provide euro and other symbols
\else % if luatex or xetex
  \usepackage{unicode-math}
  \defaultfontfeatures{Scale=MatchLowercase}
  \defaultfontfeatures[\rmfamily]{Ligatures=TeX,Scale=1}
\fi
% Use upquote if available, for straight quotes in verbatim environments
\IfFileExists{upquote.sty}{\usepackage{upquote}}{}
\IfFileExists{microtype.sty}{% use microtype if available
  \usepackage[]{microtype}
  \UseMicrotypeSet[protrusion]{basicmath} % disable protrusion for tt fonts
}{}
\makeatletter
\@ifundefined{KOMAClassName}{% if non-KOMA class
  \IfFileExists{parskip.sty}{%
    \usepackage{parskip}
  }{% else
    \setlength{\parindent}{0pt}
    \setlength{\parskip}{6pt plus 2pt minus 1pt}}
}{% if KOMA class
  \KOMAoptions{parskip=half}}
\makeatother
\usepackage{xcolor}
\IfFileExists{xurl.sty}{\usepackage{xurl}}{} % add URL line breaks if available
\IfFileExists{bookmark.sty}{\usepackage{bookmark}}{\usepackage{hyperref}}
\hypersetup{
  pdftitle={Labor Rights and Global Supply Chains},
  pdfauthor={Emmanuel Teitelbaum},
  colorlinks=true,
  linkcolor={Maroon},
  filecolor={Maroon},
  citecolor={Blue},
  urlcolor={blue},
  pdfcreator={LaTeX via pandoc}}
\urlstyle{same} % disable monospaced font for URLs
\usepackage[margin = 1in]{geometry}
\usepackage{graphicx}
\makeatletter
\def\maxwidth{\ifdim\Gin@nat@width>\linewidth\linewidth\else\Gin@nat@width\fi}
\def\maxheight{\ifdim\Gin@nat@height>\textheight\textheight\else\Gin@nat@height\fi}
\makeatother
% Scale images if necessary, so that they will not overflow the page
% margins by default, and it is still possible to overwrite the defaults
% using explicit options in \includegraphics[width, height, ...]{}
\setkeys{Gin}{width=\maxwidth,height=\maxheight,keepaspectratio}
% Set default figure placement to htbp
\makeatletter
\def\fps@figure{htbp}
\makeatother
\setlength{\emergencystretch}{3em} % prevent overfull lines
\providecommand{\tightlist}{%
  \setlength{\itemsep}{0pt}\setlength{\parskip}{0pt}}
\setcounter{secnumdepth}{-\maxdimen} % remove section numbering
\usepackage[american]{babel}
\ifLuaTeX
  \usepackage{selnolig}  % disable illegal ligatures
\fi
\usepackage[style=chicago-authordate,backend = biber,skipbib =
true,labeldateparts = true,isbn = false,annotation = false,url =
false,date = year]{biblatex}
\addbibresource{labor-rights.bib}

\begin{document}
\maketitle

\hypertarget{interaction-of-private-and-public-regulation}{%
\section{Interaction of Private and Public
Regulation}\label{interaction-of-private-and-public-regulation}}

\fullcite{Amengual2010}
\begin{quote}
Abstract: 
Although the ultimate success of labor regulation in many economic sectors depends on a combination of state and private actors, to date, researchers have not studied the interaction between state and private regulation. What happens when these forms of regulation meet on the factory floor? Based on a case study of labor inspection and code of conduct implementation in the Dominican Republic, this paper argues that the comparative advantages of state and private actors can drive complementary state\textendash private regulation. These findings suggest that private-voluntary initiatives can reinforce, rather than displace, state regulation.
\end{quote}

\emph{Summary:}

This article explores the interaction of private and public regulation
of labor standards through a case study of a garment manufacturer in the
Dominican Republic (DR). The main finding is that private regulation by
the company compliments rather than displaces regulation by the Labor
Secretariat (\emph{Secretaría de Estado de Trabajo} or SET).

The private inspections done by the company in the case study and the
state inspectorate complemented each other in a few ways. Company
inspections focused on preventing health and safety violations by
reviewing evidence of violations in the factory (e.g.~company records
and factory floor inspections) whereas the SET inspectors followed up on
reports of labor rights violations by workers, NGOs or unions. SET
inspectors were therefore less likely to uncover health and safety
violations but more helpful in addressing violations of FACB rights,
nonpayment of wages or forced overtime.

Additionally, private inspections freed up time and resources for the
SET so that it could conduct inspections of factories outside of the
EPZs, which tended to have worse labor conditions. Private inspections
also forced factories in the EPZs to comply with local labor laws,
thereby increasing the reach of the SET.

However, the analysis also uncovers differential impact of private
versus SET inspections.Factory owners were more likely to respond to
demands by the buyer to address health and safety violations because the
threat of losing the buyer's business was more compelling than the
meager fine imposed by the labor inspectorate. The company rarely cut
ties with a producer but effectively used pedagogy (training) to improve
compliance.

\bigbreak
\fullcite{Amengual2016}
\begin{quote}
Abstract: 
Research on global programs to regulate labor standards has emphasized interactions between transnational and state regulatory institutions. If transnational initiatives can make state institutions more relevant, they have the potential to reinforce, rather than displace, state labor regulation. Through a study of the Indonesia-based program of a leading initiative to improve working conditions in the garment industry, Better Work, this article identifies the conditions under which transnational regulations reinforce domestic ones. Drawing on two case studies comparing regulations governing fixed-term contracts and minimum wage renegotiations in four Indonesian districts, the authors find that reinforcement is likely when two conditions jointly occur: unions mobilize to activate state institutions, and transnational regulators have support to resolve ambiguities in formal rules in ways that require firms to engage with constraining institutions. The authors further test the findings through a quantitative analysis of factory participation in state-supervised wage renegotiations. The findings reveal opportunities and constraints to designing global programs that can both improve factory-level standards and support the functioning of state labor market institutions.
\end{quote}

\emph{Summary:}

This is a study of the Better Work program in Indonesia. It seeks to
understand how transnational regulations/institutions reinforce local
regulations/institutions. The authors argue that transnational
institutions reinforce local institutions when ``two conditions jointly
occur: unions mobilize to activate state institutions, and transnational
regulators have support to resolve ambiguities in formal rules in ways
that require firms to engage with constraining institutions.''

\hypertarget{the-alliance-and-the-accord}{%
\section{The Alliance and the
Accord}\label{the-alliance-and-the-accord}}

\fullcite{Ahlquist2021}

\begin{quote}
Abstract: 
Most research on private governance examines the design and negotiation of particular initiatives or their operation and effectiveness once established, with relatively little work on why firms join in the first place. We contribute to this literature by exploring firms' willingness to participate in two recent, high-profile private initiatives established in the aftermath of the Rana Plaza disaster in the Bangladesh ready-made garment (RMG) sector: the Accord on Building and Fire Safety and the Alliance for Worker Safety in Bangladesh. Using novel shipment-level data from U.S. customs declarations, we generate a set of firms that were ``eligible'' to join these remediation initiatives. We are able to positively attribute only a minority of US RMG imports from Bangladesh to Accord and Alliance signatories. Firms with consumer-facing brands, publicly-traded firms, and those importing more RMG product from Bangladesh were more likely to sign up for the Accord and Alliance. Firms headquartered in the USA were much less likely to sign onto remediation plans, especially the Accord.
\end{quote}

\emph{Summary:}

The paper uses customs declarations to identify what RMG firms
participate in the Accord on Building and Fire Safety and the Alliance
for Worker Safety in Bangladesh. The authors find that only a small
minority of shipments were delivered to firms that participated in the
Alliance or the Accord. These tended to be firms with customer-facing
brands, publicly traded companies and those that imported a large number
of garments from Bangladesh. Firms headquartered in the U.S. were less
likely to have signed on to either initiative but especially the
European-based Accord.

One initial reaction I had to this paper was that we sort of already
knew a lot of this. Would it have been possible to come up with a decent
estimate of the percentage of imports by Accord and Alliance firms by
looking at the sales volume of companies that are listed as signatories
to either initiative? Maybe it would still be worth doing that to get a
more global perspective.

The article contains the most extensive lit. review on private
governance initiatives imaginable. They dedicate two paragraphs pertain
to explanations for why firms join private governance initiatives. Here
are the explanations that they list:

\begin{enumerate}
\def\labelenumi{\arabic{enumi}.}
\tightlist
\item
  Public sector regulatory quality and desire to avoid regulation;
\item
  Trade and investment (e.g.~levels of participation in countries that
  constitute major trade and investment partners);
\item
  Characteristics of firms like larger firms and firms with a chief
  sustainability officer;
\item
  Consumer pressure;
\item
  Contagion effects (other firms adopting environmental initiatives);
\item
  Pressure from the NGO sector;
\item
  National corporate cultures;
\item
  Public ownership status
\end{enumerate}

\bigbreak

\fullcite{Anner2013}.

\begin{quote}
Abstract: 
The article discusses joint liability in relation to global supply chains and the causes of labor violations in international subcontracting networks as of September 2013, focusing on the Rana Plaza garment factory building collapse in Bangladesh which killed 1,129 workers. Other topics include non-governmental organizations (NGOs), industrial workplace disasters, and the global labor- and human rights-related Accord on Building and Fire Safety in Bangladesh. Jobbers agreements are examined.
\end{quote}

\emph{Summary:}

The article reviews the Accord on Fire and Building Safety in Bangladesh
(the Accord) and argues that it represents a ``new paradigm in the
enforcement of global labor and human rights.'' The Accord is special
because it recognizes the role of buyers (brands and retailers) in
producing sweatshop conditions on the ground. Rather than simply
monitoring compliance, the Accord forces brands to finance safety
upgrades through a legally binding agreement. The authors make
comparisons to ``jobbers agreements'' between workers, contractors and
lead firms in the United States that were used to combat sweatshop
conditions in the mid-20th century.

The authors argue that the root cause of sweatshops are the sprawling
subcontracting networks that have been created and maintained by lead
firms. They state that ``labor violations are not simply a factory-level
problem that can be corrected by improved compliance monitoring; they
are a pervasive and predictable outcome in an industry dominated by lead
firms whose business model is predicated on outsourcing apparel
production via highly flexible, volatile, and cost-sensitive
subcontracting networks'' (p.~3). Consequently, any viable plan for
elminating sweatshops has to address the ``root cause'' of labor
violations, which they believe is ``to be found in the sourcing
practices of the brands and retailers that coordinate these supply
chains.''

The authors provide a nice, concise summary of the literature on private
governance in formulating their critique of it. They state that previous
anlaysis can be divided into three camps essentially. The first are
those that argue that labor rights violations are the result of ``state
failure (poorly enforced or inadequate public labor regulations)'',
e.g.~Piore and Schrank. The second group of scholars emphasize
``managerial failure (lack of capacity, competence or motivation at the
factory level)'', e.g.~Richard Locke and his coauthors. The third group
of scholars focuses on the role of ``market failure (not exploiting the
niche of ethical consumers who care about labor standards).'' Here they
cite Robinson, Meyere and Kimeldorf along with Hainmueller and Hiscox
and some other studies on ethical consumerism. The authors point out the
shortcomings of each of these approaches in great detail.

Ironically, however, the authors then go on to argue that agreements
like the Accord can be brought about by ``alliances between workers,
national and international labor unions, and other actors (consumers,
student activists, etc.).'' This argument puts unions and consumer
activists in the driver's seat rather than lead firms.

With the benefit of hindsight, we can clearly see the importance of the
state and employers in the case of Bangladesh. The article by Bair et.
al.~below discusses how the Accord was basically dismantled by
politically powerful garment manufacturers who opposed it every step of
the way, and by a weak state unable to resist the pressure from these
employers. Domestic unions, international labor NGOs and lead firms were
powerless to stop the government and employers from crushing their
protest for a \$1 a day wage. One has to conclude from this that if lead
firms are serious about improving labor rights, they should not contract
from places like Bangladesh until the government and employers can get
on board with agreements like the Accord and stop repressing unions.
Consumer activists will also need to boycott goods being produced in
these countries and seek alternative modes of consumption that
circumvent these inherently exploitative supply chains.

There is a nice analysis of the relationship between prices paid to
suppliers and the decline of labor rights in this article (see Figure
2).

One really nit-picky comment: it is the Accord on Fire and Building
Safety in Bangladesh (not the ``Accord on Building and Fire Safety in
Bangladesh'').

\bigbreak

\fullcite{Bair2020}

\begin{quote}
Abstract: 
How do public and private labor governance regimes intersect in global supply chains and with what effects? Based on fieldwork in Bangladesh, including interviews with garment industry stakeholders, this article examines the main public and private regulatory reforms instituted in post-Rana Plaza Bangladesh: the Sustainability Compact and the Bangladesh Accord, respectively. Despite the Accord's substantial achievements in improving workplace safety, particularly relative to the progress achieved under the Compact, findings show that government and industry actors in Bangladesh have resisted the Accord's efforts to empower workers for fear that improved labor standards would threaten managerial control over one of the global garment industry's largest and cheapest labor forces. Rather than being an example of complementarity between private and public governance, or an effective substitution of one by the other, post-Rana Plaza Bangladesh represents an undermining of effective private regulation by a state opposed to pro-labor reforms.
\end{quote}

\emph{Summary:}

The article looks at the success of two initiatives designed to improve
worker health and safety in post-Rana Plaza Bangladesh: the
Sustainability Compact and the Bangladesh Accord. The Accord was a
progressive private regulatory initiative that emphasized co-governance
between workers and employers as well as freedom of association and
collective bargaining (FACB) rights. The Sustainability Compact was a
public regulatory initiative that covered factories not included in the
two major private regulatory initiatives in place (The Accord or the
Alliance).

The authors compare the Sustainability Compact and the Accord to test
the idea that private regulatory initiatives either complement or
substitute for public regulation. They find, instead, that the more
minimalist public initiative competed with and undermined the more
effective and progressive Accord, which was not supported by garment
manufacturers or the government.

The study is based on 61 interviews with government officials in
Bangladesh and the U.S. as well as representatives from the ILO, BGMEA,
local unions, international labor NGOs, industry experts and
stakeholders participating in the Accord and the Alliance. The
interviews were conducted between 2014 and 2019.

The authors describe the Accord as ``innovating co-governed private
regulation.'' The agreement encompassed two global trade union
federations (GTUFs) (IndustriALL and UNI) and eight Bangladeshi trade
unions, 200 apparel brands and two NGOs (Clean Clothes and WRC). The
emphasis of the Accord was on building safety but it was innovative in
that its steering committee included equal representation for labor and
companies, because of its unprecedented level of transparency and
because it enforces commitments through binding arbitration.

The Compact required the Bangladesh government to reform its labor laws
and to strengthen respect for FACB rights. The parliament approved Labor
Law Bill 2013 which included major amendments to the 2006 Bangladesh
Labor Act, but implementation was slow and enforcement lacking.
Specifically, the government left in place restrictions on organizing in
EPZs and required workers to organize 20\% of the workers in a factory
to achieve recognition (a high bar given the large size of factories).
The government also made it difficult to achieve recognition in practice
by continually rejecting valid applications. The Labor Law Bill also
undermined the provisions of the Accord which called for direct election
of workers to safety committees by leaving in place provisions that
required safety committees to be appointed by Participation Committees
(PCs), which are heavily influenced by management.

One important point the authors make is that many Bangladesh parliament
members are garment manufacturers and the garment manufacturer's
association (BGMEA) is incredibly powerful. It would be interesting to
look at this cross-nationally--how the effectiveness of public or
private regulation depends on the political influence of garment
manufacturers, which would in turn be related to the size of the garment
sector relative to a country's exports or GDP.\\
Ultimately the government and BGMEA went on the offensive against the
Accord, dooming its future prospects. In may 2018, the High Court issued
a restraining order against the Accord while the government and BGMEA
demanded that the Accord to close its Bangladesh office. An MoU was
signed that required the Accord to hand over its operations to
government-run RMG Sustainability Council and also permitted BGMEA
representatives to establish a presence in Accord offices to ensure a
smooth transition.

Note: According to
\href{https://en.wikipedia.org/wiki/Accord_on_Fire_and_Building_Safety_in_Bangladesh\#2018_Transition_Key_Accord}{Wikipedia},
the statutory body in charge of managing the Accord is called the
``Bangladesh Coordination and Remediation Cell.''

Bair et. al.'s story makes sense, but one also wonders whether the
Accord was really going to result in a meaningful defense of FACB rights
in practice absent interference from the Bangladesh government. The
emphasis of the Accord was, after all, on fire and safety and ultimately
Accord signatories did not shift their production (or even threaten to
do so) when the Bangladesh government moved to shut it down.

Subsequent to this research there was a major crackdown on unions that
were protesting for a dollar-an-hour wage (see notes on WRC report
below).

\bigbreak

\fullcite{Salminen2018}

\begin{quote}
Abstract: 
The Accord on Fire and Building Safety in Bangladesh (the Accord) is generally seen as a positive development in ensuring that Bangladeshi garment industry workers have access to safe working conditions. A central structural difference between the Accord and earlier corporate social responsibility (CSR) initiatives is that the Accord takes the form of an enforceable contract that directly connects first-world buyers with representatives of the third-world laborers of their supply chains. Traditionally, CSR mechanisms tread a fine line between a promise of decent labor conditions, often targeted at first-world consumers, and the nonbinding nature of such mechanisms, at least from the perspective of third-world laborers. The chief competitor of the Accord, the Alliance for Bangladesh Worker Safety (the Alliance), follows the traditional model. Thus the Accord represents a break from earlier nonbinding and worker-exclusive CSR by providing a new paradigm stressing enforceability and inclusivity.The novel structural aspects of the Accord are viewed positively by scholarship, interest groups, and general reporting. My starting point is this distinction between the positive, empowering image attributed to the enforceable agreement in the case of the Accord and the negative, hollow-words image of compliance mechanisms that do not take the form of an enforceable agreement, such as the Alliance. I argue that the possibilities for controlling liability allowed by an enforceable governance agreement can outweigh the possibilities for controlling liability allowed by reliance on strict conceptions of privity. From this perspective, the Accord can be critiqued as the herald of a new CSR paradigm that allows buyers new methods for controlling liability over their global supply chains. Additionally, the new paradigm comes with a whitewashing effect towards consumers and regulators. I argue that even more pronounced, however, can be its whitewashing effect towards adjudicators. Courts and arbitral tribunals may be prone to value the sanctity of the four-corners private ordering of transnational contracts, such as the Accord, over locally embedded legal safeguards.
\end{quote}

\emph{Summary:}

Two features make the Accord a unique private governance initiative: 1)
it is a legally binding and enforceable agreement; and 2) it includes
not just buyers but Bangladeshi and global trade unions. Specific
provisions include:

\begin{itemize}
\tightlist
\item
  Disclosure of suppliers and inspections of these suppliers by
  independent experts;
\item
  Public disclosure of inspection reports;
\item
  Requiring suppliers to remedy safety issues identified in the reports;
\item
  Paying suppliers prices that cover costs of repairs and renovations;
\item
  Maintaining relationships with suppliers through length of program;
\item
  Allowing democratically elected worker representatives into supplier
  factories to educate workers about workplace safety and worker rights;
\item
  Giving workers the right to refuse to work in unsafe conditions
\item
  Enabling workers to maintain income during factory downtime associated
  with renovations and repairs;
\item
  Terminating relationships with suppliers who fail to comply with
  requirements of the Accord
\end{itemize}

The Accord has been criticized for being narrow in scope (Anner et.
al.), e.g.~focusing only on worker safety and not fair labor conditions
overall, and for a lack of efficiency. But there are clear examples in
which the enforcement mechanism worked, including a \$2.3 settlement in
favor of workers.

One question I have is how the provisions of the Accord are working
after the takeover of the government-run RMG sustainability councel,
especially the independent inspections, enforcement, and allowing union
representatives in factories to educate workers.

\bigbreak

\fullcite{Zajak2017}

\begin{quote}
Abstract: 
This contribution discusses trajectories of labour power in the making. Taking a practice theory perspective on power, and focusing on the Accord on Fire and Building Safety in Bangladesh, the author asks how Bangladeshi trade unions are attempting to use changes in the industrial landscape after the factory collapse of Rana Plaza in 2013 to constitute different power sources. The article challenges assumptions in power resource theories that associational, institutional and social-cultural power are pre-existing factors, arguing that trade unions have to co-construct and enact those power sources in order for them to become meaningful. The article contributes to the debate on Networks of Labour Activism (NOLA) by showing that networked interactions with global unions and other labour support organizations help to construct power in an incremental way through information sharing, claim reframing, increasing social recognition, and the construction of a `shadow of protection' for trade unions. But it also points out new limitations resulting from managerial and political resistance, which aims to contain and reverse the growing power of labour. The Bangladesh Accord is a double-edged sword: on the one hand it provides unions with new opportunities for developing strategic capabilities, while on the other hand it is used by powerful domestic actors to discredit trade unions and mobilize workers against the constraints of the Accord.
\end{quote}

\emph{Summary:}

The article looks at how the Accord interacted with local unions in
Bangladesh. It argues that although its mandate was health and safety,
the Accord strengthened local unions in four key ways:

\begin{enumerate}
\def\labelenumi{\arabic{enumi})}
\tightlist
\item
  Sharing information--giving information from inspection reports to
  union leaders who can then use the violations as a basis for
  mobilization;
\item
  Listening to complaints--not only listening but intervening when the
  complaints are within the mandate of the Accord;
\item
  Claim reframing--sometimes violations outside of health and safety
  (such as bonus or termination of union leaders) can be reframed or
  linked to violations that are within the ambit of the Accord;
\item
  Empowering organizers when they are attacked--this is the ``shadow
  protection'' provided by the Accord.
\end{enumerate}

However, Zajak notes that reliance on the Accord can also be a reason
that union leaders are attacked because employers feel threatened by the
power of external actors. Employers are also BGMEA members and have
strong political connections to a government that is largely resistant
to the provisions of the Accord.

The article also has a good description of union fragmentation and how
it affects the structural power of organized labor. It also weighs the
benefits of external support (finances, organizational resources, etc. )
against its perils (lack of incentives for building the organizational
capacity of local unions). There is a good anecdote about IndustriALL
Bangaldesh Council organizing local affiliates in Bangladesh. Union
membership, however, is still quite low. Only three percent of factories
are organized and many shut down when they are organized.

\hypertarget{the-effectivenessflaws-of-private-regulatory-initiatives}{%
\section{The Effectiveness/Flaws of Private Regulatory
Initiatives}\label{the-effectivenessflaws-of-private-regulatory-initiatives}}

\fullcite{Anner2012}

\begin{quote}
Abstract: 

\end{quote}

\emph{Summary:}

Anner argues that corporate-sponsored inspections focus primarily on
minimal labor standards like workplace safety because they are concerned
about protecting the damage to their reputations that could stem from
violations of basic labor rights. But corporate-sponsored inspections
tend to ignore Freedom of Association (FA) rights because allowing
unions to operate would undermine corporate control over supply chains.
Anner tests this argument by coding 805 factory audits by the
\href{https://www.fairlabor.org/}{Fair Labor Association (FLA)} that
occurred between 2002 and 2010 and through an analysis of three case
studies: Russell Athletic in Honduras; Apple (Foxconn) in China; and
worker rights monitoring in Vietnam.

Anner points out that the difference between FACB and other issues is
one between rights and standards. Issues like a minimum wage, child and
overtime pay are \emph{standards} that can be addressed directly through
government regulations and stakeholder agreements. The freedom to join a
union, strike or engage in collective bargaining are \emph{rights} that
``do not dictate outcomes but guarantee procedures that mitigate the
inherent power imbalance in the employment relationship.''

Anner notes that CSR programs are vulnerable to regulatory capture, or
the ability of corporations to manipulate the agencies that are supposed
to regulate them. This happens because corporate representatives sit on
the executive boards of these programs, because the programs depend on
corporate support, because of the threat of exit (ability of
corporations to exit the program and choose a different one) and because
labor unions do not usually participate in the programs.

According to Anner, ``Labor unions are highly critical of most CSR
initiatives, arguing that the real goal is to replace not only the state
but also the union's role in defending workers' interests'' (p.~613).

Anner highlights two CSR programs in his
review--\href{https://wrapcompliance.org/}{WRAP (Worldwide Responsible
Apparel Production)} and \href{https://www.fairlabor.org/}{FLA}. WRAP
was founded by the \href{https://www.aafaglobal.org/}{American Apparel
and Footwear Association} and includes neither NGOs nor unions. The FLA
does not include unions either but has a number of important
anti-sweatshop NGOs on its board. The FLA is also one of the largest CSR
programs in the garment sector. The FLA posts all of its factory audits
\href{https://fla.fairfactories.org/web/Information/CustomReports/FactoriesReport}{online}
as part of its
\href{https://www.fairlabor.org/transparency}{transparency initiative}.
They also have a list of investigations initiated by third-party
complaints, which are mainly by unions, as part of their
\href{https://www.fairlabor.org/transparency/safeguards}{safeguards
initiative}.

Eyeballing these FLA reports, the number of audits seems to have
increased over the years. FLA randomly audits about 5\% of its
affiliate's factories each year. There are currently about 4,900 reports
available on the website. Has the composition of the audits has also
changed as well, perhaps in response to Anner's critiques, or do audits
still mostly focus on the most basic labor standards?

Starting on p.~615 Anner provides an interesting history of the
formation of the FLA and how a living wage, independent monitoring and
FA rights were considered but rejected by corporations and how unions
(UNITE) abandoned the agreement as a result.

In his analysis of FLA audits, Anner finds that FA rights violations
constitute a small minority of reported violations. The bulk of
violations fall into the categories of wages and benefits (31\%) or
health and safety (40\%). FA violations constitute just 5\% of reported
violations. Forced and child labor constitute 7\% and discrimination and
harassment 9\%. The remaining 8\% of violations fall into the category
of ``code awareness.''

Anner reasons that the small number of violations cannot reflect reality
given the high level of FA rights violations reported in other sources,
like the CIRI and Kucera labor rights indexes or state department
reports highlighting violations in particular countries. There is also a
disconnect between violations reported by third-parties and the audit
reports as well as a seeming reluctance to investigate FA violations
reported by third parties (who are mainly unions). Of third-party
complaints, FA violations constitute 32\% of all reported violations and
wages, health and safety 13\%, and wages, benefits and hours 27\%.

\textbf{Russell Athletic case study}: Russell Athletic closed a factory
in Honduras and claimed the closure was due to a decline in demand for
fleece products. Two auditors hired by the company found no written
evidence that the company closed the factory due to union activity and
placed the burden of proof on the workers in violation of ILO standards.
An ILO consultant, however, found substantial evidence of anti-union
activity and concluded that the closure was indeed a response to
unionization. FLA dismissed the work of this consultant and found in
favor of Russell Athletic. Subsequently, pressure from activists and
universities forced FLA and Russell Athletic to reverse course by
reopening the factory, rehiring the workers and placing Russell Athletic
under review.

\textbf{Vietnam and Better Work}: Anner discusses the challenge of
auditing labor standards in authoritarian countries where FA rights are
heavily restricted by law. He compares the approach of Better Work with
FLA. FLA looks for functioning workers' councils and accepts these as
substitutes for independent unions. In contrast, Better Work has a more
detailed coding scheme that separates out collective bargaining, the
right to strike and the right to join a union and finds a much higher
rater of noncompliance when it comes to FA rights (e.g.~100\%).

\textbf{Apple and Foxconn in China}: Anner analyzes FLA's audit of
Foxconn, Apple's primary supplier in China. The 197-page audit finds
numerous health and safety violations and some FA violations but ignores
some key issues. One is Apple's sourcing practices, which place extreme
demands on workers at the factory. Another is state control of unions. A
third is the local practice of electing managers to the union board,
which compensates in some ways for the workers' inability to strike.
While the report calls for an end to management control of the union, it
does not remedy the power imbalance caused by an inability to strike.

Anner discusses the Workers Rights Consortium (WRC) as an alternative to
FLA. The WRC was formed by unions and NGOs that left the FLA following
its refusal to incorporate FA rights. The WRC relies on third-party
complaints and finds a much higher level of FA rights violations in
factories that it audits, but it has a lot fewer resources that FLA.

\bigbreak

\fullcite{Baron2003}

\begin{quote}
Abstract: 
This paper introduces the subject of private politics, presents a research agenda, and provides an example involving activists and a firm. Private politics addresses situations of conflict and their resolution without reliance on the law or government. It encompasses the political competition over entitlements in the status quo, the direct competition for support from the public, bargaining over the resolution of the conflict, and the maintenance of the agreed-to private ordering. The term private means that the parties do not rely on public order, i.e., lawmaking or the courts. The term politics refers to individual and collective action in situations in which people attempt to further their interests by imposing their will on others. Four models of private politics are discussed: (1) informational competition between an activist and a firm for support from the public, (2) decisions by citizen consumers regarding a boycott, (3) bargaining to resolve the boycott, and (4) the choice of an equilibrium private ordering to govern the ongoing conflicting interests of the activist and the firm.
\end{quote}

\emph{Summary:}

The article discusses four models of private politics:

Includes an informative description of the Fair Labor Association model
(Figure 1, p.~38):

\begin{itemize}
\tightlist
\item
  Representation

  \begin{itemize}
  \tightlist
  \item
    12 companies including Nike, 21 NGOs

    \begin{itemize}
    \tightlist
    \item
      six board seats for each group
    \end{itemize}
  \item
    170 college and university affiliates

    \begin{itemize}
    \tightlist
    \item
      three board seats
    \item
      982 companies (university licensees) had applied to be affiliates
      at the time this was written
    \end{itemize}
  \item
    chair

    \begin{itemize}
    \tightlist
    \item
      one seat
    \end{itemize}
  \end{itemize}
\item
  Code governing workplace practices

  \begin{itemize}
  \tightlist
  \item
    60 hour week
  \item
    minimum wage/market wage (but no living wage)
  \item
    children at least 15 unless host government allows 14
  \item
    right to form unions
  \end{itemize}
\item
  Procedures

  \begin{itemize}
  \tightlist
  \item
    supermajority to change code (both sides can block changes)
  \item
    supermajority for selecting chair
  \end{itemize}
\item
  Independent monitoring and inspection of factories

  \begin{itemize}
  \tightlist
  \item
    Companies select the monitor from an FLA accredited list
  \item
    30\% of factories for initial 2 to 3 years; 10\% annually thereafter
  \item
    Companies are to monitor each facility annually
  \end{itemize}
\item
  Requires a plan to correct deficiencies
\item
  Reports publicly on monitoring/inspections with majority approval
\end{itemize}

Nike went above FLA standards by adopting more stringent measures
including not hiring anyone under 18, meeting U.S. air quality standards
in its factories, ending use of PVCs in its shoes and establishing
educational programs for workers in its factories.

\bigbreak

\fullcite{Locke2007}

\begin{quote}
Abstract: 
Using a unique data set based on factory audits of working conditions in over 800 of Nike's suppliers across 51 countries over the years 1998\textendash 2005, the authors explore whether monitoring for compliance with corporate codes of conduct\textemdash currently the principal way both global corporations and labor rights non-governmental organizations (NGOs) address poor working conditions in global supply chain factories\textemdash achieved remediation, as indicated by improved working conditions and stepped-up enforcement of labor rights. Despite substantial efforts and investments by Nike and its staff to improve working conditions among its suppliers, monitoring alone appears to have produced only limited results. However, when monitoring efforts were combined with other interventions focused on tackling some of the root causes of poor working conditions\textemdash in particular, by enabling suppliers to better schedule their work and to improve quality and efficiency\textemdash working conditions seem to have improved considerably.
\end{quote}

\emph{Summary:}

This is a study of Nike supplier audits. Looking primarily at the
company's management audits (M-Audits) the study finds that rule of law,
size of the factory, the length and extent of the relationship with
Nike, region and whether the supplier is a footwear manufacturer are
important predictors of audit scores. The study also finds that among
the suppliers that were audited more than once, 80\% experienced no
change or worsened over time.

The review of the debates over monitoring and labor standards is pretty
informative. Locke outlines three key discussions in the literature:

\begin{itemize}
\tightlist
\item
  Whether voluntary monitoring crowds out or complements government and
  union interventions;
\item
  Whether auditors are trustworthy and effective;
\item
  Whether the proliferation of codes of conduct undermines their
  effectiveness
\end{itemize}

\bigbreak

\fullcite{Locke2009}

\begin{quote}
Abstract: 
Private, voluntary compliance programs, promoted by global corporations and nongovernmental organizations alike, have produced only modest and uneven improvements in working conditions and labor rights in most global supply chains. Through a detailed study of a major global apparel company and its suppliers, this article argues that this compliance model rests on misguided theoretical and empirical assumptions concerning the power of multinational corporations in global supply chains, the role information (derived from factory audits) plays in shaping the behavior of key actors (e.g., global brands, transnational activist networks, suppliers, purchasing agents, etc.) in these production networks, and the appropriate incentives required to change behavior and promote improvements in labor standards in these emergent centers of global production. The authors argue that it is precisely these faulty assumptions and the way they have come to shape various labor compliance initiatives throughout the world\textemdash even more than a lack of commitment, resources, or transparency by global brands and their suppliers to these programs\textemdash that explain why this compliance-focused model of private voluntary regulation has not succeeded. In contrast, this article documents that a more commitment-oriented approach to improving labor standards coexists and, in many of the same factories, complements the traditional compliance model. This commitment-oriented approach, based on joint problem solving, information exchange, and the diffusion of best practices, is often obscured by the debates over traditional compliance programs but exists in myriad factories throughout the world and has led to sustained improvements in working conditions and labor rights at these workplaces.
\end{quote}

\emph{Summary:}

The authors identify three assumptions associated with the traditional
compliance model:

\begin{itemize}
\tightlist
\item
  Asymmetric power relations between buyers and suppliers entails that
  buyers can elicit compliance with private codes of conduct;
\item
  Audits provide accurate and unbiased information about respect for
  labor standards in supplier factories;
\item
  Buyers can provide appropriate incentives to ensure compliance
\end{itemize}

They then provide evidence of how each of these incentives is flawed
using anecdotes from interviews with suppliers and auditors of a major
apparel firm that were conducted in five countries in 2007. They note
the difficulties associated with uncovering violations in the supply
chain and how the locally stationed procurement staff of the buyer firm
are unlikely to punish factories and therefore have more leverage than
the compliance officers back at headquarters. Even if the firm does cut
off orders, this does nothing to improve labor standards because it
basically terminates the relationship with that supplier who can find
less finicky buyers.

The authors advocate a ``commitment'' approach to improving labor
standards and provide evidence of how it works from the Dominican
Republic.

One interesting statement the authors make is that the compliance model
works for some minimal standards like health and safety but is not
well-suited to enhancing FACB rights.

\bigbreak

\fullcite{Locke2010}

\begin{quote}
Abstract: 
What role can corporate codes of conduct play in monitoring compliance with international labor standards and improving working conditions in global supply chains? How does this system of private voluntary regulation relate to other strategies and regulatory approaches aimed at promoting just working conditions in global supply chains? This paper explores the potential and limitations of private voluntary regulation through a detailed matched pair case study of two factories supplying Nike, the world's largest athletic footwear and apparel company. These two factories have many similarities \textendash{} both are in Mexico, both are in the apparel industry, both produce more or less the same products for Nike (and other brands) and both are subject to the same code of conduct. On the surface, both factories appear to have similar employment (i.e. recruitment, training, remuneration) practices and they receive comparable scores when audited by Nike's compliance staff. However, underlying (and somewhat obscured by) these apparent similarities, significant differences in actual labor conditions exist between these two factories. What drives these differences in working conditions? What does this imply for traditional systems of monitoring and codes of conduct? Field research conducted at these two factories reveals that beyond the code of conduct and various monitoring efforts aimed at enforcing it, workplace conditions and labor standards are shaped by very different patterns of work organization and human resource management policies. The promotion of these alternative work/human resources management practices can complement traditional monitoring efforts in ways that promoted improved labor standards.
\end{quote}

\emph{Summary:}

This paper compares two Nike suppliers in Mexico. The authors present
the study as a most similar systems in which the two factories operate
in a very similar legal and cultural context but have different
industrial relations outcomes (which are mainly measured by wages and
job satisfaction). The management in Factory A encourages industrial
democracy and has a good relationship with the union, holds regular
meetings with workers, and deals with employee complaints in a fair and
anonymous manner. The management in Factory B is more top-down in its
approach, avoids interacting with the union and publicly humiliates
employees when they offer suggestions.

The authors do not discuss the identity of the owners of Factory A, but
Factory B is run by a Taiwanese group that imports Chinese workers.
Factory A is located in a zone with many other garment producers whereas
the owners of Factory B set it up in a far-flung green site in Western
Mexico.

The main contribution offered by the paper is to highlight the
importance of management and industrial organization for the effective
implementation of labor standards. Unfortunately there is little
discussion of codes of conduct and monitoring. The results would be more
convincing if they incorporated data from audits/inspections.

\bigbreak

\fullcite{Locke2013a}

\begin{quote}
Abstract: 
Recent research on regulation and governance suggests that a mixture of public and private interventions is necessary to improve working conditions and environmental standards within global supply chains. Yet less attention has been directed to how these different forms of regulation interact in practice. The form of these interactions is investigated through a contextualized comparison of suppliers producing for Hewlett-Packard, one of the world's leading global electronics firms. Using a unique dataset describing Hewlett-Packard's supplier audits over time, coupled with qualitative fieldwork at a matched pair of suppliers in Mexico and the Czech Republic, this study shows how private and public regulation can interact in different ways \textemdash{} sometimes as complements; other times as substitutes \textemdash{} depending upon both the national contexts and the specific issues being addressed. Results from our analysis show that private interventions do not exist within a vacuum, but rather these efforts to enforce labour and environmental standards are affected by state and non-governmental actors.
\end{quote}

\emph{Summary:}

The article compares Hewlett-Packard (HP) suppliers in Mexico and the
Czech Republic to illuminate how public and private regulation interact.
The main finding is that local contextual factors result in
complimentarity in some instances and a more contradictory relationship
in others. In Mexico, private regulatory initiatives substituted for the
absence of effective public regulation but private and public regulation
were more complementary with respect to environmental standards. In the
Czech Republic, public and private regulation complemented each other in
both arenas.

\bigbreak

\fullcite{Locke2013b}

\begin{quote}
Abstract: 
This book examines and evaluates various private initiatives to enforce fair labor standards within global supply chains. Using unique data (internal audit reports and access to more than 120 supply chain factories and 700 interviews in 14 countries) from several major global brands, including NIKE, HP and the International Labor Organization's Factory Improvement Programme in Vietnam, this book examines both the promise and the limitations of different approaches to actually improve working conditions, wages and working hours for the millions of workers employed in today's global supply chains. Through a careful, empirically grounded analysis of these programs, this book illustrates the mix of private and public regulation needed to address these complex issues in a global economy.
\end{quote}

\emph{Summary:}

\printbibliography

\end{document}
